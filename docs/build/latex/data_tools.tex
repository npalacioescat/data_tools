%% Generated by Sphinx.
\def\sphinxdocclass{report}
\documentclass[letterpaper,10pt,english]{sphinxmanual}
\ifdefined\pdfpxdimen
   \let\sphinxpxdimen\pdfpxdimen\else\newdimen\sphinxpxdimen
\fi \sphinxpxdimen=.75bp\relax

\usepackage[utf8]{inputenc}
\ifdefined\DeclareUnicodeCharacter
 \ifdefined\DeclareUnicodeCharacterAsOptional
  \DeclareUnicodeCharacter{"00A0}{\nobreakspace}
  \DeclareUnicodeCharacter{"2500}{\sphinxunichar{2500}}
  \DeclareUnicodeCharacter{"2502}{\sphinxunichar{2502}}
  \DeclareUnicodeCharacter{"2514}{\sphinxunichar{2514}}
  \DeclareUnicodeCharacter{"251C}{\sphinxunichar{251C}}
  \DeclareUnicodeCharacter{"2572}{\textbackslash}
 \else
  \DeclareUnicodeCharacter{00A0}{\nobreakspace}
  \DeclareUnicodeCharacter{2500}{\sphinxunichar{2500}}
  \DeclareUnicodeCharacter{2502}{\sphinxunichar{2502}}
  \DeclareUnicodeCharacter{2514}{\sphinxunichar{2514}}
  \DeclareUnicodeCharacter{251C}{\sphinxunichar{251C}}
  \DeclareUnicodeCharacter{2572}{\textbackslash}
 \fi
\fi
\usepackage{cmap}
\usepackage[T1]{fontenc}
\usepackage{amsmath,amssymb,amstext}
\usepackage{babel}
\usepackage{times}
\usepackage[Bjarne]{fncychap}
\usepackage[dontkeepoldnames]{sphinx}

\usepackage{geometry}

% Include hyperref last.
\usepackage{hyperref}
% Fix anchor placement for figures with captions.
\usepackage{hypcap}% it must be loaded after hyperref.
% Set up styles of URL: it should be placed after hyperref.
\urlstyle{same}
\addto\captionsenglish{\renewcommand{\contentsname}{Contents:}}

\addto\captionsenglish{\renewcommand{\figurename}{Fig.}}
\addto\captionsenglish{\renewcommand{\tablename}{Table}}
\addto\captionsenglish{\renewcommand{\literalblockname}{Listing}}

\addto\captionsenglish{\renewcommand{\literalblockcontinuedname}{continued from previous page}}
\addto\captionsenglish{\renewcommand{\literalblockcontinuesname}{continues on next page}}

\addto\extrasenglish{\def\pageautorefname{page}}

\setcounter{tocdepth}{4}
\setcounter{secnumdepth}{4}


\title{data\_tools Documentation}
\date{May 23, 2018}
\release{0.0.1}
\author{Nicolas Palacio}
\newcommand{\sphinxlogo}{\vbox{}}
\renewcommand{\releasename}{Release}
\makeindex

\begin{document}

\maketitle
\sphinxtableofcontents
\phantomsection\label{\detokenize{index::doc}}

\phantomsection\label{\detokenize{index:module-data_tools.plots}}\index{data\_tools.plots (module)}

\chapter{Module plots}
\label{\detokenize{index:data-tools-s-reference}}\label{\detokenize{index:module-plots}}
Plotting functions module.
\index{volcano() (in module data\_tools.plots)}

\begin{fulllineitems}
\phantomsection\label{\detokenize{index:data_tools.plots.volcano}}\pysiglinewithargsret{\sphinxcode{data\_tools.plots.}\sphinxbfcode{volcano}}{\emph{logfc}, \emph{logpval}, \emph{thr\_pval=0.05}, \emph{thr\_fc=2.0}, \emph{c=('C0'}, \emph{'C1')}, \emph{legend=True}, \emph{title=None}, \emph{filename=None}, \emph{figsize=None}}{}
Generates a volcano plot from the differential expression data
provided.
\begin{itemize}
\item {} \begin{description}
\item[{Arguments:}] \leavevmode\begin{itemize}
\item {} 
logfc {[}list{]}: Or any iterable type. Contains the log 
(usually base 2) fold-change values. Must have the same length
as logpval.

\item {} 
logpval {[}list{]}: Or any iterable type. Contains the -log
p-values (usually base 10). Must have the same length as
logfc.

\item {} 
thr\_pval {[}float{]}: Optional, 0.05 by default. Specifies the
p-value (non log-transformed) threshold to consider a
measurement as significantly differentially expressed.

\item {} 
thr\_fc {[}float{]}: Optional, 2. by default. Specifies the FC
(non log-transformed) threshold to consider a measurement as
significantly differentially expressed.

\item {} 
c {[}tuple{]}: Optional, (‘C0’, ‘C1’) by default (matplotlib
default colors). Any iterable containing two color arguments
tolerated by matplotlib (e.g.: {[}‘r’, ‘b’{]} for red and blue).
First one is used for non-significant points, second for the
significant ones.

\item {} 
legend {[}bool{]}: Optional, True by default. Indicates wether to
show the plot legend or not.

\item {} 
title {[}str{]}: Optional, None by default. Defines the plot
title.

\item {} 
filename {[}str{]}: Optional, None by default. If passed,
indicates the file name or path where to store the figure.
Format must be specified (e.g.: .png, .pdf, etc)

\item {} 
figsize {[}tuple{]}: Optional, None by default (default matplotlib
size). Any iterable containing two values denoting the figure
size (in inches) as {[}width, height{]}.

\end{itemize}

\end{description}

\item {} \begin{description}
\item[{Returns:}] \leavevmode\begin{itemize}
\item {} 
{[}\sphinxstyleemphasis{matplotlib.figure.Figure}{]}: Figure object containing the
volcano plot.

\end{itemize}

\end{description}

\item {} \begin{description}
\item[{Examples:}] \leavevmode
\fvset{hllines={, ,}}%
\begin{sphinxVerbatim}[commandchars=\\\{\}]
\PYG{g+gp}{\PYGZgt{}\PYGZgt{}\PYGZgt{} }\PYG{n}{volcano}\PYG{p}{(}\PYG{n}{my\PYGZus{}log\PYGZus{}fc}\PYG{p}{,} \PYG{n}{my\PYGZus{}log\PYGZus{}pval}\PYG{p}{)}
\end{sphinxVerbatim}

\noindent\sphinxincludegraphics{{volcano_example}.png}

\end{description}

\end{itemize}

\end{fulllineitems}

\phantomsection\label{\detokenize{index:module-data_tools.sets}}\index{data\_tools.sets (module)}

\chapter{data\_tools.sets}
\label{\detokenize{index:data-tools-sets}}
Set operations module.
\index{in\_all() (in module data\_tools.sets)}

\begin{fulllineitems}
\phantomsection\label{\detokenize{index:data_tools.sets.in_all}}\pysiglinewithargsret{\sphinxcode{data\_tools.sets.}\sphinxbfcode{in\_all}}{\emph{x}, \emph{N}}{}
Checks if a vector x is present in all sets contained in a list N.
\begin{itemize}
\item {} \begin{description}
\item[{Arguments:}] \leavevmode\begin{itemize}
\item {} 
x {[}tuple{]}: Or any hashable type as long as is the same
contained in the sets of N.

\item {} 
N {[}list{]}: Or any iterable type containing {[}set{]} objects.

\end{itemize}

\end{description}

\item {} \begin{description}
\item[{Returns:}] \leavevmode\begin{itemize}
\item {} 
{[}bool{]}: True if x is found in all sets of N, False otherwise.

\end{itemize}

\end{description}

\item {} \begin{description}
\item[{Examples:}] \leavevmode
\fvset{hllines={, ,}}%
\begin{sphinxVerbatim}[commandchars=\\\{\}]
\PYG{g+gp}{\PYGZgt{}\PYGZgt{}\PYGZgt{} }\PYG{n}{N} \PYG{o}{=} \PYG{p}{[}\PYG{p}{\PYGZob{}}\PYG{p}{(}\PYG{l+m+mi}{0}\PYG{p}{,} \PYG{l+m+mi}{0}\PYG{p}{)}\PYG{p}{,} \PYG{p}{(}\PYG{l+m+mi}{0}\PYG{p}{,} \PYG{l+m+mi}{1}\PYG{p}{)}\PYG{p}{\PYGZcb{}}\PYG{p}{,} \PYG{c+c1}{\PYGZsh{} \PYGZlt{}\PYGZhy{} set A}
\PYG{g+gp}{... }     \PYG{p}{\PYGZob{}}\PYG{p}{(}\PYG{l+m+mi}{0}\PYG{p}{,} \PYG{l+m+mi}{0}\PYG{p}{)}\PYG{p}{,} \PYG{p}{(}\PYG{l+m+mi}{1}\PYG{p}{,} \PYG{l+m+mi}{1}\PYG{p}{)}\PYG{p}{,} \PYG{p}{(}\PYG{l+m+mi}{1}\PYG{p}{,} \PYG{l+m+mi}{0}\PYG{p}{)}\PYG{p}{\PYGZcb{}}\PYG{p}{]} \PYG{c+c1}{\PYGZsh{} \PYGZlt{}\PYGZhy{} set B}
\PYG{g+gp}{\PYGZgt{}\PYGZgt{}\PYGZgt{} }\PYG{n}{x} \PYG{o}{=} \PYG{p}{(}\PYG{l+m+mi}{0}\PYG{p}{,} \PYG{l+m+mi}{0}\PYG{p}{)}
\PYG{g+gp}{\PYGZgt{}\PYGZgt{}\PYGZgt{} }\PYG{n}{in\PYGZus{}all}\PYG{p}{(}\PYG{n}{x}\PYG{p}{,} \PYG{n}{N}\PYG{p}{)}
\PYG{g+go}{True}
\PYG{g+gp}{\PYGZgt{}\PYGZgt{}\PYGZgt{} }\PYG{n}{y} \PYG{o}{=} \PYG{p}{(}\PYG{l+m+mi}{0}\PYG{p}{,} \PYG{l+m+mi}{1}\PYG{p}{)}
\PYG{g+gp}{\PYGZgt{}\PYGZgt{}\PYGZgt{} }\PYG{n}{in\PYGZus{}all}\PYG{p}{(}\PYG{n}{y}\PYG{p}{,} \PYG{n}{N}\PYG{p}{)}
\PYG{g+go}{False}
\end{sphinxVerbatim}

\end{description}

\end{itemize}

\end{fulllineitems}

\index{bit\_or() (in module data\_tools.sets)}

\begin{fulllineitems}
\phantomsection\label{\detokenize{index:data_tools.sets.bit_or}}\pysiglinewithargsret{\sphinxcode{data\_tools.sets.}\sphinxbfcode{bit\_or}}{\emph{a}, \emph{b}}{}
Returns the bit operation OR between two bit-strings a and b.
NOTE: a and b must have the same size.
\begin{itemize}
\item {} \begin{description}
\item[{Arguments:}] \leavevmode\begin{itemize}
\item {} 
a {[}tuple{]}: Or any iterable type.

\item {} 
b {[}tuple{]}: Or any iterable type.

\end{itemize}

\end{description}

\item {} \begin{description}
\item[{Returns:}] \leavevmode\begin{itemize}
\item {} 
{[}tuple{]}: OR operation between a and b element-wise.

\end{itemize}

\end{description}

\item {} \begin{description}
\item[{Examples:}] \leavevmode
\fvset{hllines={, ,}}%
\begin{sphinxVerbatim}[commandchars=\\\{\}]
\PYG{g+gp}{\PYGZgt{}\PYGZgt{}\PYGZgt{} }\PYG{n}{a}\PYG{p}{,} \PYG{n}{b} \PYG{o}{=} \PYG{p}{(}\PYG{l+m+mi}{0}\PYG{p}{,} \PYG{l+m+mi}{0}\PYG{p}{,} \PYG{l+m+mi}{1}\PYG{p}{)}\PYG{p}{,} \PYG{p}{(}\PYG{l+m+mi}{1}\PYG{p}{,} \PYG{l+m+mi}{0}\PYG{p}{,} \PYG{l+m+mi}{1}\PYG{p}{)}
\PYG{g+gp}{\PYGZgt{}\PYGZgt{}\PYGZgt{} }\PYG{n}{bit\PYGZus{}or}\PYG{p}{(}\PYG{n}{a}\PYG{p}{,} \PYG{n}{b}\PYG{p}{)}
\PYG{g+go}{(1, 0, 1)}
\end{sphinxVerbatim}

\end{description}

\end{itemize}

\end{fulllineitems}

\index{multi\_union() (in module data\_tools.sets)}

\begin{fulllineitems}
\phantomsection\label{\detokenize{index:data_tools.sets.multi_union}}\pysiglinewithargsret{\sphinxcode{data\_tools.sets.}\sphinxbfcode{multi\_union}}{\emph{N}}{}
Returns the union set of all sets contained in a list N.
\begin{itemize}
\item {} \begin{description}
\item[{Arguments:}] \leavevmode\begin{itemize}
\item {} 
N {[}list{]}: Or any iterable type containing {[}set{]} objects.

\end{itemize}

\end{description}

\item {} \begin{description}
\item[{Returns:}] \leavevmode\begin{itemize}
\item {} 
{[}set{]}: The union of all sets contained in N.

\end{itemize}

\end{description}

\item {} \begin{description}
\item[{Examples:}] \leavevmode
\fvset{hllines={, ,}}%
\begin{sphinxVerbatim}[commandchars=\\\{\}]
\PYG{g+gp}{\PYGZgt{}\PYGZgt{}\PYGZgt{} }\PYG{n}{A} \PYG{o}{=} \PYG{p}{\PYGZob{}}\PYG{l+m+mi}{1}\PYG{p}{,} \PYG{l+m+mi}{3}\PYG{p}{,} \PYG{l+m+mi}{5}\PYG{p}{\PYGZcb{}}
\PYG{g+gp}{\PYGZgt{}\PYGZgt{}\PYGZgt{} }\PYG{n}{B} \PYG{o}{=} \PYG{p}{\PYGZob{}}\PYG{l+m+mi}{0}\PYG{p}{,} \PYG{l+m+mi}{1}\PYG{p}{,} \PYG{l+m+mi}{2}\PYG{p}{\PYGZcb{}}
\PYG{g+gp}{\PYGZgt{}\PYGZgt{}\PYGZgt{} }\PYG{n}{C} \PYG{o}{=} \PYG{p}{\PYGZob{}}\PYG{l+m+mi}{0}\PYG{p}{,} \PYG{l+m+mi}{2}\PYG{p}{,} \PYG{l+m+mi}{5}\PYG{p}{\PYGZcb{}}
\PYG{g+gp}{\PYGZgt{}\PYGZgt{}\PYGZgt{} }\PYG{n}{multi\PYGZus{}union}\PYG{p}{(}\PYG{p}{[}\PYG{n}{A}\PYG{p}{,} \PYG{n}{B}\PYG{p}{,} \PYG{n}{C}\PYG{p}{]}\PYG{p}{)}
\PYG{g+go}{set([0, 1, 2, 3, 5])}
\end{sphinxVerbatim}

\end{description}

\end{itemize}

\end{fulllineitems}

\index{find\_min() (in module data\_tools.sets)}

\begin{fulllineitems}
\phantomsection\label{\detokenize{index:data_tools.sets.find_min}}\pysiglinewithargsret{\sphinxcode{data\_tools.sets.}\sphinxbfcode{find\_min}}{\emph{A}}{}
Finds and returns the subset of vectors whose sum is minimum from a
given set A.
\begin{itemize}
\item {} \begin{description}
\item[{Arguments:}] \leavevmode\begin{itemize}
\item {} 
A {[}set{]}: Set of vectors ({[}tuple{]} or any iterable).

\end{itemize}

\end{description}

\item {} \begin{description}
\item[{Returns:}] \leavevmode\begin{itemize}
\item {} 
{[}set{]}: Subset of vectors in A whose sum is minimum.

\end{itemize}

\end{description}

\item {} \begin{description}
\item[{Examples:}] \leavevmode
\fvset{hllines={, ,}}%
\begin{sphinxVerbatim}[commandchars=\\\{\}]
\PYG{g+gp}{\PYGZgt{}\PYGZgt{}\PYGZgt{} }\PYG{n}{A} \PYG{o}{=} \PYG{p}{\PYGZob{}}\PYG{p}{(}\PYG{l+m+mi}{0}\PYG{p}{,} \PYG{l+m+mi}{1}\PYG{p}{,} \PYG{l+m+mi}{1}\PYG{p}{)}\PYG{p}{,} \PYG{p}{(}\PYG{l+m+mi}{0}\PYG{p}{,} \PYG{l+m+mi}{1}\PYG{p}{,} \PYG{l+m+mi}{0}\PYG{p}{)}\PYG{p}{,} \PYG{p}{(}\PYG{l+m+mi}{1}\PYG{p}{,} \PYG{l+m+mi}{0}\PYG{p}{,} \PYG{l+m+mi}{0}\PYG{p}{)}\PYG{p}{,} \PYG{p}{(}\PYG{l+m+mi}{1}\PYG{p}{,} \PYG{l+m+mi}{1}\PYG{p}{,} \PYG{l+m+mi}{1}\PYG{p}{)}\PYG{p}{\PYGZcb{}}
\PYG{g+gp}{\PYGZgt{}\PYGZgt{}\PYGZgt{} }\PYG{n}{find\PYGZus{}min}\PYG{p}{(}\PYG{n}{A}\PYG{p}{)}
\PYG{g+go}{set([(0, 1, 0), (1, 0, 0)])}
\end{sphinxVerbatim}

\end{description}

\end{itemize}

\end{fulllineitems}

\phantomsection\label{\detokenize{index:module-data_tools.strings}}\index{data\_tools.strings (module)}

\chapter{data\_tools.strings}
\label{\detokenize{index:data-tools-strings}}
String operations module.
\index{is\_numeric() (in module data\_tools.strings)}

\begin{fulllineitems}
\phantomsection\label{\detokenize{index:data_tools.strings.is_numeric}}\pysiglinewithargsret{\sphinxcode{data\_tools.strings.}\sphinxbfcode{is\_numeric}}{\emph{s}}{}
Determines if a string can be considered a numeric value.
NaN is also considered, since it is float type.
\begin{itemize}
\item {} \begin{description}
\item[{Arguments:}] \leavevmode\begin{itemize}
\item {} 
s {[}str{]}: String to be evaluated.

\end{itemize}

\end{description}

\item {} \begin{description}
\item[{Returns:}] \leavevmode\begin{itemize}
\item {} 
{[}bool{]}: True/False depending if the condition is satisfied.

\end{itemize}

\end{description}

\item {} \begin{description}
\item[{Examples:}] \leavevmode
\fvset{hllines={, ,}}%
\begin{sphinxVerbatim}[commandchars=\\\{\}]
\PYG{g+gp}{\PYGZgt{}\PYGZgt{}\PYGZgt{} }\PYG{n}{is\PYGZus{}numeric}\PYG{p}{(}\PYG{l+s+s1}{\PYGZsq{}}\PYG{l+s+s1}{4}\PYG{l+s+s1}{\PYGZsq{}}\PYG{p}{)}
\PYG{g+go}{True}
\PYG{g+gp}{\PYGZgt{}\PYGZgt{}\PYGZgt{} }\PYG{n}{is\PYGZus{}numeric}\PYG{p}{(}\PYG{l+s+s1}{\PYGZsq{}}\PYG{l+s+s1}{\PYGZhy{}3.2}\PYG{l+s+s1}{\PYGZsq{}}\PYG{p}{)}
\PYG{g+go}{True}
\PYG{g+gp}{\PYGZgt{}\PYGZgt{}\PYGZgt{} }\PYG{n}{is\PYGZus{}numeric}\PYG{p}{(}\PYG{l+s+s1}{\PYGZsq{}}\PYG{l+s+s1}{number}\PYG{l+s+s1}{\PYGZsq{}}\PYG{p}{)}
\PYG{g+go}{False}
\PYG{g+gp}{\PYGZgt{}\PYGZgt{}\PYGZgt{} }\PYG{n}{is\PYGZus{}numeric}\PYG{p}{(}\PYG{l+s+s1}{\PYGZsq{}}\PYG{l+s+s1}{NaN}\PYG{l+s+s1}{\PYGZsq{}}\PYG{p}{)}
\PYG{g+go}{True}
\end{sphinxVerbatim}

\end{description}

\end{itemize}

\end{fulllineitems}

\index{join\_str\_lists() (in module data\_tools.strings)}

\begin{fulllineitems}
\phantomsection\label{\detokenize{index:data_tools.strings.join_str_lists}}\pysiglinewithargsret{\sphinxcode{data\_tools.strings.}\sphinxbfcode{join\_str\_lists}}{\emph{a}, \emph{b}, \emph{sep=''}}{}
Joins element-wise two lists (or any 1D iterable) of strings with a
given separator (if provided). Length of the input lists must be
equal.
\begin{itemize}
\item {} \begin{description}
\item[{Arguments:}] \leavevmode\begin{itemize}
\item {} 
a {[}list{]}: Contains the first elements {[}str{]} of the joint
strings.

\item {} 
b {[}list{]}: Contains the second elements {[}str{]} of the joint
strings.

\item {} 
sep {[}str{]}: Optional ‘’ (non separated) by default. Determines
the separator between the joint strings.

\end{itemize}

\end{description}

\item {} \begin{description}
\item[{Returns:}] \leavevmode\begin{itemize}
\item {} 
{[}list{]}: List of the joint strings.

\end{itemize}

\end{description}

\item {} \begin{description}
\item[{Example:}] \leavevmode
\fvset{hllines={, ,}}%
\begin{sphinxVerbatim}[commandchars=\\\{\}]
\PYG{g+gp}{\PYGZgt{}\PYGZgt{}\PYGZgt{} }\PYG{n}{a} \PYG{o}{=} \PYG{p}{[}\PYG{l+s+s1}{\PYGZsq{}}\PYG{l+s+s1}{a}\PYG{l+s+s1}{\PYGZsq{}}\PYG{p}{,} \PYG{l+s+s1}{\PYGZsq{}}\PYG{l+s+s1}{b}\PYG{l+s+s1}{\PYGZsq{}}\PYG{p}{]}
\PYG{g+gp}{\PYGZgt{}\PYGZgt{}\PYGZgt{} }\PYG{n}{b} \PYG{o}{=} \PYG{p}{[}\PYG{l+s+s1}{\PYGZsq{}}\PYG{l+s+s1}{1}\PYG{l+s+s1}{\PYGZsq{}}\PYG{p}{,} \PYG{l+s+s1}{\PYGZsq{}}\PYG{l+s+s1}{2}\PYG{l+s+s1}{\PYGZsq{}}\PYG{p}{]}
\PYG{g+gp}{\PYGZgt{}\PYGZgt{}\PYGZgt{} }\PYG{n}{join\PYGZus{}str\PYGZus{}lists}\PYG{p}{(}\PYG{n}{a}\PYG{p}{,} \PYG{n}{b}\PYG{p}{,} \PYG{n}{sep}\PYG{o}{=}\PYG{l+s+s1}{\PYGZsq{}}\PYG{l+s+s1}{\PYGZus{}}\PYG{l+s+s1}{\PYGZsq{}}\PYG{p}{)}
\PYG{g+go}{[\PYGZsq{}a\PYGZus{}1\PYGZsq{}, \PYGZsq{}b\PYGZus{}2\PYGZsq{}]}
\end{sphinxVerbatim}

\end{description}

\end{itemize}

\end{fulllineitems}



\chapter{Indices and tables}
\label{\detokenize{index:indices-and-tables}}\begin{itemize}
\item {} 
\DUrole{xref,std,std-ref}{genindex}

\item {} 
\DUrole{xref,std,std-ref}{modindex}

\item {} 
\DUrole{xref,std,std-ref}{search}

\end{itemize}


\renewcommand{\indexname}{Python Module Index}
\begin{sphinxtheindex}
\def\bigletter#1{{\Large\sffamily#1}\nopagebreak\vspace{1mm}}
\bigletter{d}
\item {\sphinxstyleindexentry{data\_tools.plots}}\sphinxstyleindexpageref{index:\detokenize{module-data_tools.plots}}
\item {\sphinxstyleindexentry{data\_tools.sets}}\sphinxstyleindexpageref{index:\detokenize{module-data_tools.sets}}
\item {\sphinxstyleindexentry{data\_tools.strings}}\sphinxstyleindexpageref{index:\detokenize{module-data_tools.strings}}
\end{sphinxtheindex}

\renewcommand{\indexname}{Index}
\printindex
\end{document}